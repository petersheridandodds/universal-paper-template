%% editor: y.-y. ahn
%% sune lehmann
%% neil johnson
%% martin gerlach
%% simon dideo
%% aaron clauset
%% dashun wang
%% luis amaral

To the Editors at EPJ Data Science,

We are very pleased to submit our manuscript for consideration at
EPJ Data Science:

\textbf{``Allotaxonometry and rank-turbulence divergence: A universal instrument for comparing complex systems.''}

We introduce and frame the concept of allotaxonometry:
The comparison of complex systems which have broad
internal variation in component type and structure.

We argue against the mismeasurement of complexity by single
numbers, that we must instead put forward instruments and tools
that embrace complexity.
Just as we need dashboards to fly planes,
we need dynamic dashboards for comprehending complex systems in general.

We start from the longstanding but misleading observation that Zipf distributions
for language may vary little across texts.
We show that while the overall form of Zipf's distribution may be conserved,
the ordering of words (or n-grams) can vary greatly, a phenomenon we have
previously dubbed `lexical turbulence'.

For general complex systems,
we build a general `allotaxonometric instrument' that shows a map-like
histogram for Zipf distribution changes along with a ranked list
of the most `important' components according to whatever instrument
we use to measure and hence distinguish two systems (e.g., Shannon's entropy).

In our manuscript, we introduce `rank-turbulence divergence', a pragmatic, tunable
instrument for comparing any pair of complex systems where components can
be ranked according to some criterion
(e.g., frequency of words or species, sizes of cites or companies).

We provide a series of four main case studies in the paper, all comparing
systems evolving in time:
Word usage on Twitter,
species abundance,
baby name popularity,
and
market capitalization for publically traded companies in the US.

In online supplementary material, we expand on these
system-system comparions
with pdf flipbooks, and add more examples including
changes over time for causes of death in Hong Kong
and advertised job titles in the US.

See: \url{http://compstorylab.org/allotaxonometry/}.

Our paper is the first in a series of related papers which will
present allotaxonographs for probability-based measures such
as the Jensen-Shannon Divergence (and generalizations) as well
as our own probability-turbulence divergence.
As is, the figure-making code contains all of these options.

We note that we are in fact separately submitting our
second paper to EPJ Data Science:
\textbf{``Probability-turbulence divergence: A tunable
allotaxonometric instrument for comparing
heavy-tailed categorical distributions,''}
which is available online at \url{https://arxiv.org/abs/2008.13078}).


%% \textbf{1.}
%% \textbf{2.}
%% \textbf{3.}
%% \textbf{4.}

%% \textbf{}

We believe our paper will be of great benefit to a broad academic audience.

Finally, some possible reviewers we have listed are:
Simon Dideo,
Sune Lehman,
Y.-Y. Ahn,
Yu-Ru Lin,
Neil Johnson,
and
Johan Ugander.

We also suggest:
Jake Hofman,
Martin Gerlach,
Dashun Wang,
and
Luis Amaral.

We look forward to hearing of your decision.

\bigskip
\bigskip

Yours sincerely and on behalf of the manuscript's authors,


%% Professor
%% Director of the Vermont Complex Systems Center
%% Co-Director, Computational Story Lab
%% Vermont Advanced Computing Center
%% Department of Mathematics and Statistics
%% The University of Vermont
